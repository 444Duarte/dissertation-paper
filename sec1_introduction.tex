\section{Introduction}
Continuing the current trend, mobile data usage is expected to keep increasing
exponentially, part of it due to mobile video streaming and
IoT. The estimation is that the number of data that was generated
by mobile devices during the year of 2017 exceeded the $\displaystyle 6 * 10^9$ Gb
per month. Together with the traffic generated by laptops and peer-to-peer
communications, overall traffic demand might reach $\displaystyle 11 * 10^9$ Gb per
month\cite{kn:Dehos2014} \cite{kn:Baresi2017}. To compute such a big amount of
data, cloud computing would appear to be the obvious solution but there are cases
where the latency that comes with transmitting data back and forth might be
a limitation. In certain situations, it is also not feasible to expect a constant and
reliable internet connection to an always-on server, either because it might not
be economically wise or because it might not be infrastructurally possible. In
order to solve the need for low latency, as well as to improve fault tolerance, by
not relying on an always-on centralized server, serverless architectures and fog
computing aim to reduce the dependency on the cloud by making more use of the
local resources of a device and improving communication between local devices,
only leaving the data-intensive tasks to the cloud\cite{kn:Baresi2017}. With the
increasing trend in serverless solutions, such as AWS Lambda, it is opportunistic
to implement these concepts in IoT.

Although IoT has been around for a few years already, the same cannot be said
about services that provide cloud solutions and cloud infrastructure for rent. 
Likewise, when Serverless and Fog Computing solutions first appeared their
usefulness and benefits for the IoT ecosystem was obvious and developers began to
mix them together in order to get the most out of this new trend.

Despite its success and the promising future for the mix of this concepts, the
area is still fairly new and few solutions can take advantage of the processing
power in the cloud and in the local network of IoT devices in an efficient way
without compromising speed. It is already possible to have a network of IoT
devices working together to execute a series of serverless functions, but not all
serverless functions are suitable to run on low-end devices. To choose where each
serverless functions should be executed (locally or in the cloud) is a manual task
and the end result is that developers choose to have all serverless functions
running in the cloud as it is easier to manage and less risky. Nonetheless, there
is a lot of potential processing power dormant in each local network that could
be used to improve response times, improve fault-tolerance,  and to slash costs of
hosting cloud processing infrastructure. 
