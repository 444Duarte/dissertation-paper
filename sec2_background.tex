\section{Background}

\subsection{Internet of Things}\label{sec:dialecto}
The \textit{Internet of Things} is a term given to the network of ever increasing
number of mundane objects with embedded systems, that allows them to interact with
each other or with someone remotely. This network creates a smarter and
self-regulated environment that depends less on the input of a physical entity and
more on the input of other \textit{things}, so that it removes unnecessary steps
and it frees the user from dealing with mundane tasks.

\subsection{Serverless}
Defining the term serverless can be difficult, as the term is both misleading and
its definition overlaps other concepts, such as Platform-as-a-Service (PaaS) and
Software-as-a-Service(SaaS). Serverless stands in between these two concepts, where
the developer loses some control over the cloud infrastructure but maintains
control over the application code \cite{kn:Baldini}.

"The term ‘Serverless’ is confusing since with such applications there are both
server hardware and server processes running somewhere, but the difference to
normal approaches is that the organization building and supporting a ‘Serverless’ application is not looking after the hardware or the processes - they are outsourcing this to a vendor." 

\hfill Mike Roberts

\hfill 2016

The most important area of serverless, for this paper, is
\textbf{Function-as-a-Service (FaaS)} in which, the server-side logic is still
written and controlled by the developers, but they run in stateless containers
that are triggered by events, ephemeral and are fully managed by the 3rd party
entity. Despite being a recent paradigm, there is already some investigation about
the performance and usability of serverless solutions managed by 3rd party
entities \cite{kn:Lee}.

\subsection{Fog Computing}
Fog Computing is a virtual resource paradigm, located in between the Cloud
layer(traditional cloud or data centers) and the Edge layer (smart end devices) in
order to provide computing power, storage, and networking services. Although
conceptually located in between the two layers, in practice, this platform is located at the
edge of the network \cite{kn:Bonomi}. "This paradigm supports vertically-isolated,
latency-sensitive applications by providing ubiquitous, scalable, layered,
federated, and distributed computing, storage, and network connectivity"~\cite{kn:Iorga2017}. 


