\section{Problem Statement}

As stated before in previous sections, not only is expected for the number of IoT
devices to grow immensely, both commercially and industrially, but there are
already many solutions that allow for serverless functions to be executed
remotely. Due to the nature of serverless functions, some of them could perfectly
be executed locally, using the joint processing power of the multiple IoT devices.
The hardship comes with using this power efficiently, having multiple serverless
functions, and knowing where to execute each one, locally or remotely. It is not
feasible for each developer to manually analyze performance across the different
runtime environments and make a decision where the function should be executed.
This is impeding the adoption of these concepts in IoT, despite the interest and
potential that exists in this evergrowing area. Not only there is a lack of
systems making use of serverless on-premises, the majority of the developers in
IoT opt for using the cloud for each and every need, disregarding the power that
exists locally.

Like what was presented above, there is a lack of practical know-how knowledge
available despite there being lots of incentives for it. There are lots of things,
but it is not easy to start developing a serverless IoT solution.

Given this, the aim of this project is to create an architecture for serverless IoT
platform and to build a proof of concept using existing open-source tools when
possible and avoiding proprietary solutions. The platform should:
\begin{itemize}
    \item Have a serverless cloud solution capable of answering HTTP requests from
        the \textit{things}.
    \item Make use of the local processing power of the multiple IoT devices to
        create a serverless virtual processing unit on the Fog Layer to answer local requests.
    \item Have multiple IoT devices with different functions capable of
        interacting with both the Cloud and Fog layer to execute different functions. 
\end{itemize}


